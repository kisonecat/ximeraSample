\documentclass{ximera}

\newcommand{\mypreamble}{WHEEEEE}
%%% Local Variables:
%%% mode: latex
%%% TeX-master: t
%%% End:

\usepackage{currfile}

\makeatletter
\newcommand{\makerandom}{%
  \ST@wsf{jobname="\currfilebase"}%
  \ST@wsf{import hashlib}%
  \ST@wsf{set_random_seed(int(hashlib.sha256(jobname.encode('utf-8')).hexdigest(), 16))}%
  \ST@wsf{latex.vector_delimiters('\\langle', '\\rangle')}
}
\makeatother


\author{Jim Fowler}

\license{Creative Commons 3.0 By-SA}
\outcome{Compute the distance between a given point and a line.}

\begin{document}

\makerandom

\begin{sagesilent}
  x = var('x')
  y = var('y')

  p = q = v = w = nearest_point = vector([0,0,0])

  while nearest_point.norm() <= 1:
    p = vector([randint(-1,1),randint(-2,2),randint(-1,1)])
    v = vector([randint(-1,1),randint(-2,2),randint(-1,1)])
    q = v.dot_product(v)*vector([randint(-1,1),randint(-1,1),randint(-1,1)]) + p
    w = q-p
    nearest_point = ((q - p).dot_product(v) / v.dot_product(v)) * v + p
\end{sagesilent}

\begin{exercise}
  Consider the line $\vecl$ in the direction $\vec{v} = \vector{\sage{v[0]},\sage{v[1]},\sage{v[2]}}$ which passes through the point $\sage{p}$.
  
  How far is the point \(\sage{q}\) from the line \(\vecl\)?  It is
  \(\answer{\sage{(q - nearest_point).norm()}}\) units.

  \begin{hint}
    We can rephrase this problem in terms of vectors.
  \end{hint}

  \begin{hint}
    The tip of the vector $\vec{p} = \vector{\sage{p[0]},\sage{p[1]},\sage{p[2]}}$ is on the line $\vecl$.
  \end{hint}

  \begin{hint}
    The vector $\vec{w} = \vector{\sage{(q-p)[0]},\sage{(q-p)[1]},\sage{(q-p)[2]}}$ starts at the point $\sage{p}$ and ends at the point $\sage{q}$.
  \end{hint}

  \begin{hint}
    The length of the cross product $\vec{w} \cross \vec{v}$ is $|\vec{v}|$ times the distance from the point to the line.
  \end{hint}

  \begin{hint}
    In this case, $\vec{w} \cross \vec{v} = \sage{w.cross_product(v)}$.
  \end{hint}

  \begin{hint}
    Consequently, the distance from the point $\sage{p}$ to the line $\vecl$ is the length of $\sage{w.cross_product(v)}$ divided by $|\vec{v}| = \sage{v.norm()}$.  This turns out to be $\sage{w.cross_product(v).norm()/v.norm()}$ units.
  \end{hint}

  \begin{hint}
    With more work (say, using orthogonal projection), we could also discover that $\sage{nearest_point}$ is the point on $\vecl$ nearest to the point $\sage{q}$.
  \end{hint}
  
\end{exercise}

\end{document}
