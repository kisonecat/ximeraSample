\documentclass{ximera}
\usepackage{cancel}
\usepackage{currfile}

\makeatletter
\ifxake
% The code below the \else is executed in a sagecell on the Ximera
% server, so \makerandom doesn't have to do anything when run under
% xake.
\newcommand{\makerandom}{}
\else
\newcommand{\makerandom}{%
  \ST@wsf{jobname="\currfilebase"}%
  \ST@wsf{import hashlib}%
  \ST@wsf{set_random_seed(int(hashlib.sha256(jobname.encode('utf-8')).hexdigest(), 16))}%
}
\fi
\makeatother


\title[Breal-grond]{This is the sample! Waffle!  Iff!}

\begin{document}
\begin{abstract}
  The begining and now an instructor! I am doing math like $\arcsin$. AGAIN
\end{abstract}

\maketitle
\makerandom

compare $\mathrm{x}$ to x.

compare $M$ to \textit{M}.

But also if $y^2 = 4y$ then if $y \neq 0$ we may compute $\cancel{y} \cdot y = 4 \cancel{y}$ so $y = 4$.

Look at this:
\[
  \pm 2
\]

I roll a D20 and I get $\sage{randint(1,20)}$.

I roll a D1000 and I get $\sage{randint(1,1000)}$.

\marginpar{Pretty fun huh?}

\begin{sagesilent}
  lhs = randint(0,10)
  rhs = randint(0,10)
  x = rhs - lhs
\end{sagesilent}

I can concatenate with $cat + dog = \sagestr{'cat' + 'dog'}$.

\begin{problem}
  Suppose $\sage{lhs} + x = \sage{rhs}$.

  Then $x = \answer{\sage{x}}$.
\end{problem}

It is a simple exercise to solve the system of two equations
\begin{equation} \label{small}
\begin{array}{rcrcr}
 x & + & y & = & 7 \\
-x & + & 3y & = & 1
\end{array}
\end{equation}
to find that $x=5$ and $y=2$.  One way to solve
system \eqref{small} is to add the two equations, obtaining
\[
4y=8;
\]
hence $y=2$.  Substituting $y=2$ into the $1^{st}$ equation in
\ref{small} yields $x=5$.


%I am testing the waffle and the iff statement.

%\begin{tikzpicture}
%  \draw (0,0) circle (1in);
%  \draw (0,0) circle (0.25in);
%  \draw (0,0) -- (1in,1in);
%\end{tikzpicture}

\youtube{eGXGlG5CuYE}

and another one!

\youtube{eGXGlG5CuYE}


We might look at \ref{thm:another} but consider \ref{thm:whee}.

\begin{theorem}
  \label{thm:whee}  Whee theorem.
\end{theorem}
\begin{proof}
This is the proof for reals.
\end{proof}

\end{document}


% Changed at Tue Jul 28 00:06:03 EDT 2015
% Changed at Tue Jul 28 00:06:04 EDT 2015
% Changed at Tue Jul 28 00:06:28 EDT 2015
