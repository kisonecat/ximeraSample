\documentclass{ximera}

\title[Breal-grond]{This is the sample! Waffle!  Iff!}

\begin{document}
\begin{abstract}
  The begining and now an instructor! I am doing math like $\arcsin$. AGAIN
\end{abstract}

\maketitle

compare $\mathrm{x}$ to x.

compare $M$ to \textit{M}.

Look at this:
\[
  \pm 2
\]

\begin{sagesilent}
  x = var('x')
  p = x*x - x + 1
\end{sagesilent}

On the other hand, $12 + 5 = \left( \sage{12+5} \right)$ which is computed.

Likewise,
\[
  x^{12} \cdot (1 + x^{12}) = \left(\sage{x^12 + x^24}\right) = \sage{factor(x^12 + x^24)}.
\]

It is a simple exercise to solve the system of two equations
\begin{equation} \label{small}
\begin{array}{rcrcr}
 x & + & y & = & 7 \\
-x & + & 3y & = & 1
\end{array}
\end{equation}
to find that $x=5$ and $y=2$.  One way to solve
system \eqref{small} is to add the two equations, obtaining
\[
4y=8;
\]
hence $y=2$.  Substituting $y=2$ into the $1^{st}$ equation in
\ref{small} yields $x=5$.


%I am testing the waffle and the iff statement.

%\begin{tikzpicture}
%  \draw (0,0) circle (1in);
%  \draw (0,0) circle (0.25in);
%  \draw (0,0) -- (1in,1in);
%\end{tikzpicture}

%\youtube{eGXGlG5CuYE}

and another one!

% \youtube{eGXGlG5CuYE}


We might look at \ref{thm:another} but consider \ref{thm:whee}.

\begin{theorem}
  \label{thm:whee}  Whee theorem.
\end{theorem}
\begin{proof}
This is the proof for reals.
\end{proof}

\end{document}


% Changed at Tue Jul 28 00:06:03 EDT 2015
% Changed at Tue Jul 28 00:06:04 EDT 2015
% Changed at Tue Jul 28 00:06:28 EDT 2015
