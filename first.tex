\documentclass{ximera}


%\usepackage{currfile}

%\makeatletter
%\ifxake
% The code below the \else is executed in a sagecell on the Ximera
% server, so \makerandom doesn't have to do anything when run under
% xake.
%\newcommand{\makerandom}{}
%\else
%\newcommand{\makerandom}{%
%  \ST@wsf{jobname="\currfilebase"}%
%  \ST@wsf{import hashlib}%
%  %\ST@wsf{set_random_seed(int(hashlib.sha256(jobname.encode('utf-8')).hexdiges%t(), 16))}%
%}
%\fi
%\makeatother


\title[Breal-grond]{This is the sample! Waffle!  Iff!}

\begin{document}
\begin{abstract}
  The begining and now an instructor! I am doing math like $\arcsin$. AGAIN
\end{abstract}

\maketitle
%\makerandom

HELLO.

\begin{javascriptCode}
x = 0;
y = 0;

vx = 0;
vy = 0;

deltaVx = 0;
deltaVy = 0;

theta = 0;

Fthrust = 30.0;
mass = 3.0;
dt = 0.1;

p.draw = function() {
    // Update velocities
    vx += deltaVx;

    // Update location
    x += vx*dt;

    // velocity is unchanged if there are no forces
    deltaVx = 0;

    // Turn or thrust the ship depending on what key is pressed
    if (keyIsDown(LEFT_ARROW)) {
        theta += 0.0;
    }
    if (keyIsDown(RIGHT_ARROW)) {
        theta += 0.0;
    }
    if (keyIsDown(UP_ARROW)) {
        // Rockets on!
        accelx = Fthrust*cos(theta)/mass;
        deltaVx = accelx*dt;
    }
    if (keyIsDown(DOWN_ARROW)) {
            // do nothing
    }
    if (keyIsPressed && key == ' '){ //spacebar
        // Do nothing!
    }
    // Draw ship and other stuff
    // This will clear the screen and re-draw it
    display();
    // Add more graphics here before the end of draw()
  
} // end draw()

////////////////////////////////////////////////////////////////

xhistory = [];
yhistory = [];

p.setup = function() {
    createCanvas(750, 500);
    x = width/2;
    y = height/2;
}

// Size of the ship
var r = 12;

var showarrows = true;

iterations = 0;

// Draw the ship and other stuff
function display() {
    wrapEdges();
    background(255);
    textSize(12);
    textStyle(NORMAL);

//    stroke(0);
    strokeWeight(10);
//    println("lets do this");
    var tri_width=7;
    if (showarrows) {
            var x_line=10;
            var y_line=25;
            var line_len=100;
            drawLine(x_line,y_line,x_line,y_line+line_len);
            drawLine(x_line,y_line,x_line+line_len,y_line);
        fill(0);
        drawTriangle(x_line-tri_width/2,y_line+line_len,x_line+tri_width/2,y_line+line_len,x_line,y_line+line_len+10);
        drawTriangle(x_line+line_len,y_line-tri_width/2,x_line+line_len,y_line+tri_width/2,x_line+line_len+10,y_line);
            strokeWeight(0);
            drawText("+x",x_line+line_len+15,y_line);
            drawText("+y",x_line,y_line+line_len+15);
    }

    if (showarrows) {
    textSize(20);
    strokeWeight(0);
    fill(255,0,0);
    text("Velocity",0.8*width,0.8*height+25);
    fill(0,0,255);
    text("Force",0.8*width,0.8*height+50);
    fill(204,0,204);
    text("Acceleration",0.8*width,0.8*height+75);
    }

    ship(x,y,vx,vy,deltaVx/dt,deltaVy/dt,theta);

  	strokeWeight(0);
    textSize(12);
    text("left right arrows to turn, tap up arrow to thrust, press H to hide the arrows, press U to un-hide",10,10);

      if (iterations%5 == 1) {  
    append(xhistory,x);
    append(yhistory,y);
    }

    iterations += 1;   
  
    if (keyIsPressed) {
     isrunning = true; 
    }
  
    MaxLength = 50;
  	if (xhistory.length > MaxLength) {
    xhistory = subset(xhistory,xhistory.length-MaxLength,xhistory.length);
    yhistory = subset(yhistory,yhistory.length-MaxLength,yhistory.length);  
    }  
  
    fill(0,0,0); //If more text is written elsewhere make sure the default is black
    stroke(0,0,0); // If more lines are drawn elsewhere make sure the default is black
    strokeWeight(0);
  
}

function wrapEdges() {
    var buffer = r*2;
    if (x > width +  buffer) x = -buffer;
    else if (x <    -buffer) x = width+buffer;
    if (y > height + buffer) y = -buffer;
    else if (y <    -buffer) y = height+buffer;
}

function drawBlob( _x,  _y){
    strokeWeight(2);
    //    fill(255);
    noFill();
    stroke(0);
    ellipse(_x, height - _y, 50, 50);  
}

function drawBlob( _x,  _y, _vx, _vy){
    strokeWeight(2);
    //    fill(255);
    noFill();
    stroke(0);
    ellipse(_x, height - _y, 50, 50);  
  
            strokeWeight(10);
    var tri_width=7;

    // Draw velocity arrow
    var v_scaling=5.0;
    stroke(255,0,0); // makes the line red
    strokeWeight(3); // makes the line thicker

    if ( ((_vx !== 0) || (_vy !== 0)) && showarrows) {
        drawLine(_x,_y,_x+v_scaling*_vx,_y+v_scaling*_vy);
        var vel_angle = -atan2(_vy,_vx);
        fill(255,0,0); // makes the triangle red
        drawTriangle(_x+v_scaling*_vx+sin(vel_angle)*tri_width/2,_y+v_scaling*_vy+cos(vel_angle)*tri_width/2,_x+v_scaling*_vx-sin(vel_angle)*tri_width/2,_y+v_scaling*_vy-cos(vel_angle)*tri_width/2,_x+v_scaling*_vx+cos(vel_angle)*10,_y+v_scaling*_vy-sin(vel_angle)*10);
    }
  


      fill(0,0,0); //If more text is written elsewhere make sure the default is black
    stroke(0,0,0); // If more lines are drawn elsewhere make sure the default is black
    strokeWeight(0);

}



function ship( _x,  _y, _vx, _vy, _ax, _ay, _theta)
{
    strokeWeight(2);
    //    fill(255);
    noFill();
    stroke(0);

    stroke(0);
    strokeWeight(2);
    push();
    translate(_x,height-_y);
    rotate(-theta+PI/2);
    fill(175);
    // A triangle
    beginShape();
    vertex(-r,r);
    vertex(0,-1.5*r);
    vertex(r,r);
    endShape(CLOSE);
    rectMode(CENTER);
    pop();
    fill(0);
    strokeWeight(0);
  
  
  
    strokeWeight(10);
    var tri_width=7;

    // Draw velocity arrow
    var v_scaling=1.0;
    stroke(255,0,0); // makes the line red
    strokeWeight(3); // makes the line thicker

    if ( ((_vx !== 0) || (_vy !== 0)) && showarrows) {
        drawLine(_x,_y,_x+v_scaling*_vx,_y+v_scaling*_vy);
        var vel_angle = -atan2(_vy,_vx);
        fill(255,0,0); // makes the triangle red
        drawTriangle(_x+v_scaling*_vx+sin(vel_angle)*tri_width/2,_y+v_scaling*_vy+cos(vel_angle)*tri_width/2,_x+v_scaling*_vx-sin(vel_angle)*tri_width/2,_y+v_scaling*_vy-cos(vel_angle)*tri_width/2,_x+v_scaling*_vx+cos(vel_angle)*10,_y+v_scaling*_vy-sin(vel_angle)*10);
    }

     // Draw force arrow
    var f_scaling=2.25;
//    var f_scaling=5.0;
    var Fx = mass*_ax;
    var Fy = mass*_ay;
    var f_angle = -atan2(Fy,Fx);

    if (((Fx !== 0) || (Fy !== 0)) && showarrows) {
//    if (((Fx != 0) || (Fy != 0)) && 0 ) {
    stroke(0,0,255); // makes the line blue
    drawLine(_x,_y,_x+f_scaling*Fx,_y+f_scaling*Fy);
    fill(0,0,255); // makes the triangle blue
    drawTriangle(_x+f_scaling*Fx+sin(f_angle)*tri_width/2,_y+f_scaling*Fy+cos(f_angle)*tri_width/2,_x+f_scaling*Fx-sin(f_angle)*tri_width/2,_y+f_scaling*Fy-cos(f_angle)*tri_width/2,_x+f_scaling*Fx+cos(f_angle)*10,_y+f_scaling*Fy-sin(f_angle)*10);    
    }
  
    var a_scaling=2.25;
    f_angle = -atan2(_ay,_ax);
    if (((_ax !== 0) || (_ay !== 0)) && showarrows) {
    stroke(204,0,204); // makes the line purple
    drawLine(_x,_y,_x+a_scaling*_ax,_y+a_scaling*_ay);
    fill(204,0,204); // makes the triangle purple
    drawTriangle(_x+a_scaling*_ax+sin(f_angle)*tri_width/2,_y+a_scaling*_ay+cos(f_angle)*tri_width/2,_x+a_scaling*_ax-sin(f_angle)*tri_width/2,_y+a_scaling*_ay-cos(f_angle)*tri_width/2,_x+a_scaling*_ax+cos(f_angle)*10,_y+a_scaling*_ay-sin(f_angle)*10);
    }
  

      fill(0,0,0); //If more text is written elsewhere make sure the default is black
    stroke(0,0,0); // If more lines are drawn elsewhere make sure the default is black
    strokeWeight(0);

}


/*
function drawBlob( _x,  _y, _r){
  strokeWeight(2);
  ellipse(_x, height - _y, _r, _r);  
}*/

function drawEllipse( _x,  _y,  _w,  _h){
  ellipse(_x, height - _y, _w, _h);  
}

function drawLine( _x1,  _y1,  _x2,  _y2){
  strokeWeight(3);
  line(_x1, height - _y1, _x2, height - _y2);  
//  strokeWeight(0);
}

function drawPoint( _x,  _y){
    strokeWeight(3);
    point(_x, height - _y);  
    strokeWeight(0);
}

function drawQuad( _x1,  _y1,  _x2,  _y2,  _x3,  _y3,  _x4,  _y4){
  quad(_x1, height - _y1, _x2, height - _y2, _x3, height - _y3, _x4, height - _y4);  
}

function drawRect( _x,  _y,  _w,  _h){
  rect(_x, height - _y, _w, _h);  
}

function drawRect( _x,  _y,  _w,  _h,  _r){
  rect(_x, height - _y, _w, _h, _r);  
}

function drawRect( _x,  _y,  _w,  _h,  _tl,  _tr,  _br,  _bl){
  rect(_x, height - _y, _w, _h, _tl, _tr, _br, _bl);  
}

function drawTriangle( _x1,  _y1,  _x2,  _y2,  _x3,  _y3){
  triangle(_x1, height - _y1, _x2, height - _y2, _x3, height - _y3);
}

function drawText( _str,  _x, _y){
    if (isNumeric(_str)){
        _str = round(100*_str)/100;
    }
    textSize(20);  
    strokeWeight(1);
    text(_str, _x, height- _y);
}

function isNumeric(n) {
    return !isNaN(parseFloat(n)) && isFinite(n);
}

\end{javascriptCode}

compare $\mathrm{x}$ to x.

compare $M$ to \textit{M}.

But also if $y^2 = 4y$ then if $y \neq 0$ we may compute $\cancel{y} \cdot y = 4 \cancel{y}$ so $y = 4$.

$21 \cancel{\textrm{ feet}}$

Look at this:
\[
  \pm 2
\]

%I roll a D20 and I get $\sage{randint(1,20)}$.

%I roll a D1000 and I get $\sage{randint(1,1000)}$.

\marginpar{Pretty fun huh?}

%\begin{sagesilent}
%  lhs = randint(0,10)
%  rhs = randint(0,10)
%  x = rhs - lhs
%\end{sagesilent}

%I can concatenate with $cat + dog = \sagestr{'cat' + 'dog'}$.

%\begin{problem}
%  Suppose $\sage{lhs} + x = \sage{rhs}$.

%  Then $x = \answer{\sage{x}}$.
%\end{problem}

It is a simple exercise to solve the system of two equations
\begin{equation} \label{small}
\begin{array}{rcrcr}
 x & + & y & = & 7 \\
-x & + & 3y & = & 1
\end{array}
\end{equation}
to find that $x=5$ and $y=2$.  One way to solve
system \eqref{small} is to add the two equations, obtaining
\[
4y=8;
\]
hence $y=2$.  Substituting $y=2$ into the $1^{st}$ equation in
\ref{small} yields $x=5$.


I am testing the waffle and the iff statement.

%\begin{tikzpicture}
%  \draw (0,0) circle (1in);
%  \draw (0,0) circle (0.25in);
%  \draw (0,0) -- (1in,1in);
%\end{tikzpicture}

\youtube{eGXGlG5CuYE}

and another one!

\youtube{eGXGlG5CuYE}


We might look at \ref{thm:another} but consider \ref{thm:whee}.

\begin{theorem}
  \label{thm:whee}  Whee theorem.
\end{theorem}
\begin{proof}
This is the proof for reals.
\end{proof}

\end{document}


% Changed at Tue Jul 28 00:06:03 EDT 2015
% Changed at Tue Jul 28 00:06:04 EDT 2015
% Changed at Tue Jul 28 00:06:28 EDT 2015
