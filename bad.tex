\documentclass[handout,nooutcomes]{ximera}
%handout
%wordchoicegiven
%space
%nooutcomes
\title{Math 160 Lab 5}
\author{The Moomaster} 
\newcommand{\mypreamble}{WHEEEEE}
%%% Local Variables:
%%% mode: latex
%%% TeX-master: t
%%% End:


\outcome{Explain what is meant by the `arc length' of a function.}
\outcome{Describe how to approximate the arc length of a function on a specified interval using a fixed number of line segments.}
\outcome{Give the definition of a smooth function and determine if a particular function is smooth on an interval.}
\outcome{State a formula that gives the arc length of a smooth function on a specified interval.}
\outcome{Determine the arc length of a smooth function using the arc length formula.}
\outcome{Find the arc length of a function that is smooth with respect to y but not x.}
\outcome{Find the arc length of a function that is not smooth at its endpoints but has symmetry.}

\begin{document}

\section{Calculus 1 Lab 5 \\ An Application of Integration - Arc Length}

%% Have to edit the date here each semester.
\begin{abstract}
In this lab, we will explore how we can calculate arc length using a definite integral.  This lab corresponds to section 6.3 in our physical textbook, so feel free to use the textbook as a resource while completing this lab.\\

Unless stated otherwise, input answers in \underline{exact form} on this lab.
\end{abstract}
 
\maketitle


\begin{problem}

Determine the arc length of $f(x) = \sin(x)$ on the interval $[0, 2\pi].$

$f'(x) = \answer{\cos(x)}$

\begin{problem}

Therefore $f(x)$ is \wordChoice {\choice[correct]{smooth}\choice{not smooth}} on $[0,2\pi]$ because \wordChoice {\choice{f(x) is continuous} \choice[correct]{f'(x) is continuous} \choice{f(x) is not continuous} \choice{f'(x) is not continuous}} on $[0,2\pi]$.  

\begin{problem}

Finally, we can write an integral to evaluate the arc length of the curve on the specified interval and evaluate it using technology.  Our original approximation of this arc length with $n=4$ line segments was 7.45, which turns out to be pretty close to the exact value.  

Arc Length = $\displaystyle\int_{\answer{0}}^{\answer{2\pi}} \sqrt{1+{\answer{\cos(x)}^2}} \ dx \approx \answer{7.64}$ (Evaluate the arc length to 2 decimal places with technology)


\end{problem}
\end{problem}
\end{problem}

Let's try calculating the arc length of another function.  This time, you will be able to evaluate the resulting integral for arc length by hand using the techniques we've learned in class.

%%Practice Problem 2 - Standard WRT x%%

\begin{problem}

Find the exact arc length of $y = x^{\frac{3}{2}}$ from $x=0$ to $x=4$.  First, we determine if $y$ is smooth on the x-interval $[0,4]$.  

$y' = \answer{\frac{3}{2}\sqrt{x}},$ so $y$ is \wordChoice{\choice[correct]{smooth}\choice{not smooth}} on $[0,4]$ because \wordChoice {\choice{y is continuous} \choice[correct]{y' is continuous} \choice{y is not continuous} \choice{y' is not continuous}} on $[0,4]$.  

\begin{problem}

Arc Length = $\displaystyle\int_{\answer{0}}^{\answer{4}} \answer{\sqrt{1+\frac{9x}{4}}} \ dx = \answer{\frac{8}{27}(10^{\frac{3}{2}}-1)} \ \approx 9.07$ (Evaluate the arc length exactly)

\end{problem}
\end{problem}

The formula for arc length that we derived earlier also applies to functions of $y$ as long as one integrates with respect to y, rather than x.  

\begin{theorem}[Formula for Arc Length of Smooth Functions (of y)]
If $g(y)$ is a smooth function on the y-interval $[c,d],$ then the arc length of $g(y)$ on $[c,d]$ is equal to 

$$\displaystyle\int_{c}^{d} \sqrt{1+(g'(y))^2} \ dy$$

\end{theorem}

Here's an example.  

%%Practice Problem 3 - WRT y%%

\begin{problem}
Find the arc length of $x = \frac{2}{3} (y-1)^{\frac{3}{2}}$ from $y=1$ to $y=4$.  As usual, we must first check that this function of $y$ is smooth on the $y$-interval $[1,4]$.  

$x' = \answer{\sqrt{y-1}},$ so $x$ is \wordChoice{\choice[correct]{smooth}\choice{not smooth}} on the y-interval $[1,4]$.

\begin{problem}
Therefore, because $x$ is smooth, we can determine its arc length on $[1,4]$.

Arc Length = $\displaystyle\int_{\answer{1}}^{\answer{4}} \answer{\sqrt{y}} \ d\answer{y} = \answer{\frac{14}{3}}$ (Evaluate the arc length exactly)

\end{problem}

\end{problem}

Sometimes even if a function $f(x)$ is not smooth, there are still ways to use the arc length formula we derived, as the last few examples will illustrate.

%%Practice Problem 4 - Integrating WRT y When $f$ isn't smooth WRT x%%

\begin{problem}
Find the arc length of $f(x) = \left(\frac{x}{2}\right)^{\frac{2}{3}}$ on $[0,2]$.  

As usual, let's check if $f(x)$ is smooth on $[0,2]$ to determine if the arc length formula applies.

$f'(x) = \answer{(\frac{1}{3})(\frac{2}{x})^{\frac{1}{3}}}.$ 

Therefore, $f(x)$ is \wordChoice {\choice{smooth}\choice[correct]{not smooth}} on $[0,2]$ because \wordChoice {\choice{y is continuous} \choice{y' is continuous} \choice{y is not continuous} \choice[correct]{y' is not continuous}} on $[0,2]$.  

\begin{problem}
In this case, $f'(0)$ does not exist, so $f(x)$ is smooth on $(0,2]$ but not $[0,2]$.  Since $f(x)$ is not a smooth function of $x$ on the interval $[0,2]$, we cannot use the arc length formula in terms of $x$.  However, we can still determine the arc length if $f(x)$ can be written as a smooth function of $y$.  Let's try this and see if we get lucky.  

Written as a function of $y$, $f$ becomes $x = \answer{2y^{\frac{3}{2}}}$ and the $y$-interval that corresponds with the $x$-interval $[0,2]$ is $[0, \answer{1}]$.  

\begin{problem}
From this, we see that $x' = \answer{3\sqrt{y}}$, so $x$ is \wordChoice {\choice[correct]{smooth}\choice{not smooth}} on $[0,1]$.  

\begin{problem}
Now we can use the arc length formula (\textbf{with respect to y}) in order to evaluate the desired arc length of $f(x).$

Arc Length = $\displaystyle\int_{\answer{0}}^{\answer{1}} \answer{\sqrt{1+9y}} \ d\answer{y} = \answer{\frac{2}{27}(10^{\frac{3}{2}}-1)} \ \approx 2.27$ (Evaluate the arc length exactly)

\end{problem}
\end{problem}
\end{problem}
\end{problem}

%%Practice Problem 5- Using symmetry when $f$ isn't smooth WRT x%%

Let's try one last arc length problem involving a non-smooth function.  Let's determine the exact arc length of $f(x) = \sqrt{9-x^2}$ on the interval $[-3,3]$, the graph of which is displayed below.


\begin{problem}
First off, notice that $f(x)$ is not smooth on $[-3,3]$ because (select all that apply)
\begin{selectAll}
    \choice{f(-3) does not exist.}
    \choice{f(3) does not exist.}
    \choice{f is not continuous at $x=-3$.}
    \choice{f is not continuous at $x=3$.}
    \choice[correct]{f'(-3) does not exist.}
    \choice[correct]{f'(3) does not exist.}
    \choice{f' is not continuous on $(-3,3)$.}
    \choice[correct]{f' is not continuous at the endpoints of $[-3,3]$.}
\end{selectAll}
\end{problem}

Since $f$ isn't smooth on $[-3,3]$, the arc length formula does not apply.  Why not?  Well, $f'(x) = -\frac{x}{\sqrt{9-x^2}}$, so if the arc length formula did apply, we would need to evaluate 

$$\displaystyle\int_{-3}^{3} \sqrt{1+\frac{x^2}{9-x^2}} \ dx.$$ 

The integrand of the above integral has the following graph on the interval $[-3,3].$  \\


Notice that this integrand has vertical asymptotes at $x=-3$ and $x=3$, so there is no guarantee that the area bounded between $\sqrt{1+\frac{x^2}{9-x^2}}$ and the $x$-axis will converge to a particular number.  In other words, it's possible that this integral cannot be evaluated, despite the fact that the arc length of $f(x)$ can certainly be calculated - $f(x)$ is the graph of a semi-circle after all.  This is one instance of why the arc length formula can only be applied when $f(x)$ is smooth on $[-3,3]$.  Luckily, because $f$ is semi-circular, we do not need calculus to find its arc length.

\begin{problem}
Use the fact that $f(x) = \sqrt{9-x^2}$ is the graph of a semi-circle to evaluate the arc length of $f$ on $[-3,3]$ without using calculus.  

The arc length is exactly $\answer{3\pi}.$
\end{problem}

\end{document}
