\documentclass{ximera}
\title{Another waffle}
\begin{document}
\begin{abstract}
  This is another thing.  Waffle and fi and fine.
\end{abstract}
\maketitle

Here is a waffle.  And fine.  And fish.

Here we go.

\begin{problem}
  But $3 + 3 = \answer{6}$ and yet $x + x = \answer{2x}$.
\end{problem}

\begin{foldable}
\begin{problem}
  Compute $5 + 5 \begin{prompt}=\answer{10}\end{prompt}$.
\end{problem}
\end{foldable}

Here is a picture for us to test.

\begin{image}
  \begin{tikzpicture}
    \draw (0,0) circle (1in);
  \end{tikzpicture}
\end{image}
 
And what about multiple choices?
\begin{problem}
\begin{multipleChoice}
  \choice{Wrong!}
  \choice{Not right.}
  \choice[correct]{Yes, pick me.}
  \choice{Not me either.}
\end{multipleChoice}
\end{problem}

I am testing this ---
 thing with \ref{a-label}.

Or try \ref{good-problem}.

\begin{theorem}
  \label{thm:another}
\end{theorem}

\begin{corollary}
  \label{thm:whee} is repeated.
\end{corollary}

\end{document}

%%% Local Variables:
%%% mode: latex
%%% TeX-master: t
%%% End:
% Changed at Mon Jul 27 22:23:04 EDT 2015
% Changed at Mon Jul 27 22:24:20 EDT 2015
% Changed at Mon Jul 27 22:25:23 EDT 2015

