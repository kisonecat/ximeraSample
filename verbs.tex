\documentclass{ximera}
 
\title{Writing exercises}
 
\begin{document}
\begin{abstract}
  We give a workflow and template for exercises.
\end{abstract}
\maketitle
 
\section{Template for exercises}
 
Exercises are fairly easy to write. All exercises should be in a
directory named ``\verb|exercises|'' that is contained within the
directory for the section to which the exercises pertain. For an
example, see
\link[here]{https://github.com/mooculus/calculus/tree/master/workingInTwoAndThreeDimensions}
 
The exercises should have outcomes that align with the outcomes from
the section and title page. Authors of exercises should check that
these outcomes align.
 
\begin{verbatim}
\documentclass{ximera}
 
\author{Jane Doe}
 
\outcome{Some outcome from the chapter.}
 
\newcommand{\mypreamble}{WHEEEEE}
%%% Local Variables:
%%% mode: latex
%%% TeX-master: t
%%% End:

 
\begin{document}
\begin{exercise}
What is $2+2$?
\[
2+2 = \answer{4}
\]
\end{exercise}
\end{document}
\end{verbatim}
 
\section{Keeping track of exercises}
 
It is important to keep track of all exercises. We do this with
various exercise lists.
 
In the \verb|exercises| directory, one should make a file
\verb|exerciseList.tex| that lists the exercises. As an example, see \link[this]{https://github.com/mooculus/calculus/blob/master/workingInTwoAndThreeDimensions/exercises/exerciseList.tex}.
 
Additionally, one should add there exercises to a larger exercise list,
found in the \verb|exercises| directory, at the top-level of the
calculus repository. See \link[this]{https://github.com/mooculus/calculus/tree/master/exercises}.
 
 
 
\end{document}
