\documentclass{ximera}

\begin{document}

We can randomly present problems:

\begin{shuffle}
  \begin{problem}
    The first answer is $\answer{1} = 1$
  \end{problem}

  \begin{problem}
    The second answer is $\answer{2} = 2$
  \end{problem}

  \begin{problem}
    The third answer is $\answer{3} = 3$
  \end{problem}

  \begin{problem}
    The fourth answer is $\answer{4} = 4$

  \end{problem}
\end{shuffle}

\begin{problem}
  The tolerance (0.1) means $8 \approx \answer[tolerance=0.1]{8}$
\end{problem}


YouTube:
\youtube{ED-oYu4e3LU}

GeoGebra:
\geogebra{222093}{711}{697}

Desmos:
\desmos{agpxst6ewq}{200}{200}

JavaScript:
\begin{html}
<script src="https://ajax.googleapis.com/ajax/libs/jquery/1.11.3/jquery.min.js"></script>
<script src='jsDed.js'></script>
<h1 id="time" style="background-color:lightgray">Mouseover for the time.</h1>
\end{html}

\begin{problem}
  The tolerance (0.01) means $8 \approx \answer[tolerance=0.01]{8}$
\end{problem}

\begin{problem}
  The tolerance (0.01) means $\pi \approx \answer[tolerance=0.01]{3.141592653}$
\end{problem}

\begin{problem}
  The tolerance (17) means $3421 \approx \answer[tolerance=17]{3421}$
\end{problem}

\begin{code}
  var Expression = require('math-expressions');

  Expression.fromText('x^2');
\end{code}

\begin{verbatim}
  Hello!  This is verbatim.
\end{verbatim}

\begin{problem}
  We can add such as in $2 + 2 = \answer{4}$.
  
  \begin{problem}
    Multiplication looks like $3 \times 3 = \answer{9}$.
    
    \begin{problem}
      Now consider $\sqrt{\answer{16}} = 4$.
      
      \begin{problem}
        Therefore $4 \times 4 = \answer{16}$.
      \end{problem}
    \end{problem}
  \end{problem}
\end{problem}

\begin{theorem}
Look at my shadow!
\end{theorem}

\begin{problem}
  \begin{multipleChoice}
    \choice{Incorrect}
    \choice{Wrong}
    \choice[correct]{It's this one}
    \choice{Not Right}
  \end{multipleChoice}

  \begin{problem}
    There were $\answer{4}$ possible answers to that question.

    \begin{problem}
      \begin{multipleChoice}
        \choice{Not correct}
        \choice[correct]{Pick me!}
        \choice{False}
        \choice{Untrue}
      \end{multipleChoice}
    \end{problem}
  \end{problem}
\end{problem}

\end{document}

%%% Local Variables:
%%% mode: latex
%%% TeX-master: t
%%% End:
% Changed at Mon Jul 27 22:23:04 EDT 2015
% Changed at Mon Jul 27 22:24:20 EDT 2015
% Changed at Mon Jul 27 22:25:23 EDT 2015
% Changed at Mon Jul 27 22:30:13 EDT 2015
% Changed at Mon Jul 27 22:32:16 EDT 2015
% Changed at Mon Jul 27 22:33:48 EDT 2015
% Changed at Tue Jul 28 00:05:31 EDT 2015
% Changed at Tue Jul 28 00:05:33 EDT 2015
% Changed at Tue Jul 28 00:05:34 EDT 2015
% Changed at Tue Jul 28 00:05:35 EDT 2015
% Changed at Tue Jul 28 00:05:55 EDT 2015
% Changed at Tue Jul 28 00:05:56 EDT 2015
% Changed at Tue Jul 28 00:05:57 EDT 2015
% Changed at Tue Jul 28 00:05:58 EDT 2015
% Changed at Tue Jul 28 00:05:59 EDT 2015
% Changed at Tue Jul 28 00:06:00 EDT 2015
% Changed at Tue Jul 28 00:06:01 EDT 2015
% Changed at Tue Jul 28 00:06:03 EDT 2015
% Changed at Tue Jul 28 00:06:04 EDT 2015
% Changed at Tue Jul 28 00:06:28 EDT 2015
