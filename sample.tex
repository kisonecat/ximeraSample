\documentclass{ximera}

\title{Sample interactives}
\author{Jim Fowler}

\begin{document}
\begin{abstract}
Testing interactives and ff.
\end{abstract}
\maketitle

Now I'm using metadata too.

TJHIRD

\begin{theorem}\label{thm1}
  THIS IS a theorem
\end{theorem}

There's a free response box here.
\begin{freeResponse}
  
\end{freeResponse}

\begin{problem}\label{problema}
  Blah!
  \begin{prompt}
    The answer equals $\answer{3}$
  \end{prompt}  
\end{problem}

\begin{problem}
  There are two things here.\label{problemb}
  \begin{ungraded}
    \begin{multipleChoice}
      \choice{Cats}
      \choice{Dogs}      
    \end{multipleChoice}
  \end{ungraded}
\end{problem}

\begin{problem}\label{problemc}
  \wordChoice{\choice{$x \neq x$}\choice{$1 = 2$}\choice[correct]{YES}}  
\end{problem}

\begin{theorem}\label{thm-again}
  THIS IS another theorem.
\end{theorem}

\begin{problem}\label{problemd}
  There are two things here.
  \begin{multipleChoice}
    \choice[correct]{correct!}
    \choice{wrong.}      
  \end{multipleChoice}
\end{problem}

\begin{proposition}
  THIS IS a proposition\label{propo}
\end{proposition}


\begin{problem}
  No, really, my favorite number is $y = \answer[format=integer,id=y]{17}$.

  \begin{feedback}[attempt]
    You made a first attempt!
  \end{feedback}

  \begin{feedback}[y>17]
    That number is TOO BIG.
  \end{feedback}

  \begin{feedback}[y<17]
    That number is too small.
  \end{feedback}

  \begin{feedback}[correct]
    I have always loved the number 17.
  \end{feedback}
\end{problem}

\begin{javascript}
  function isPositive(number) {
    return number > 0;
  };

  function sameParity(a,b) {
    return (a-b)%2 == 0;
  };

  caseInsensitive = function(a,b) {
    return a.toLowerCase() == b.toLowerCase();
  };

  sameDerivative = function(a,b) {
    return a.derivative('x').equals( b.derivative('x') );
  };

  slowOdd = function(a) {
    return new Promise( function(resolve, reject) {
      if (a == 0)
        reject('I do not like zero.');
      else
        setTimeout(function(){
          resolve(a % 2 == 1);
        }, 1000);      
    });
  };  
\end{javascript}


\begin{problem}
  Pick an odd integer like $\answer[format=integer,validator=slowOdd]{11}$.
\end{problem}

So $1 < 2 > 0$ is a test of math and all.

\begin{sageCell}
z = var('z')
polynomial = z*z + 1
\end{sageCell}

\begin{problem}
  Pick an integer larger than 10, like $\answer[id=big,format=integer,validator=this>10]{11}$.  A bigger integer is \js{big+1}.

  Pick $\answer[format=integer,validator=isPositive]{17} > 0$.

  Find a solution to $\answer[id=odd,format=integer,validator=sameParity]{5} \equiv 5 \bmod 2$.  In that case, $\js{odd} - 5 = \js{odd-5} \equiv 0 \bmod 2$.

  String equality is like $\mbox{Word} = \answer[format=string]{Word}$, but if you wanted to see a case-insensitive comparison try $\mbox{Word} = \answer[format=string,validator=caseInsensitive]{Word}$.

  Find a function $f(x) = \answer[validator=sameDerivative]{x^2}$ so that $f'(x) = \frac{d}{dx} x^2$.
\end{problem}

\begin{problem}
  Here I am making a change.
  
  Pick an integer $\answer[id=x,format=integer,validator=true]{17}$.

  Then solve $\js{x} + 1$ equals $\answer[format=integer,validator=this==x+1]{18}$.
\end{problem}

\begin{problem}
  Find two integers which sum to twenty five.

  \begin{validator}[a+b==25]
    $\answer[format=integer,id=a]{16} + \answer[format=integer,id=b]{9} = 25$.
  \end{validator}


  Find a traceless matrix with integer entries.
  \begin{validator}[a11+a22==0]
    \[
      \begin{bmatrix}
        \answer[format=integer,id=a11]{0} &         \answer[format=integer,id=a12]{0} \\
        \answer[format=integer,id=a21]{0} &         \answer[format=integer,id=a22]{0} \\
      \end{bmatrix}
      \]
  \end{validator}

\end{problem}




\end{document}
Wed Aug  9 07:45:49 EDT 2017
Wed Aug  9 17:53:52 EDT 2017
Wed Aug  9 17:55:33 EDT 2017
Fri Aug 11 14:22:32 EDT 2017
