\documentclass{ximera}

\newcommand{\mypreamble}{WHEEEEE}
%%% Local Variables:
%%% mode: latex
%%% TeX-master: t
%%% End:

\usepackage{currfile}

\makeatletter
\newcommand{\makerandom}{%
  \ST@wsf{jobname="\currfilebase"}%
  \ST@wsf{import hashlib}%
  \ST@wsf{set_random_seed(int(hashlib.sha256(jobname.encode('utf-8')).hexdigest(), 16))}%
  \ST@wsf{latex.vector_delimiters('\\langle', '\\rangle')}
}
\makeatother


\author{Jim Fowler}

\outcome{Compute the partial derivative of an expression.}

\begin{document}
\makerandom

\begin{sagesilent}
  vs = ['x','y','z','t','w']
  shuffle(vs)
  x = var(vs[0])
  y = var(vs[1])
  z = var(vs[2])
  t = var(vs[3])  

  def random_expression(depth):
    if depth == 0:
      return [x,x,x,y,z,t,1,-1][randint(0,7)]
    else:
      possible = [
        lambda: random_expression(depth-1) + random_expression(depth-1),
        lambda: random_expression(depth-1) - random_expression(depth-1),
        lambda: random_expression(depth-1) * random_expression(depth-1),
        lambda: random_expression(depth-1) / random_expression(depth-1),
        lambda: sin(random_expression(depth-1)),
        lambda: cos(random_expression(depth-1)),
        lambda: log(random_expression(depth-1)),
        lambda: exp(random_expression(depth-1))
      ]
      result = possible[randint(0,len(possible)-1)]()
      if result.is_zero():
        result = 1
      return result

  f = random_expression(3)
  while (derivative(f,x).is_zero()) or (len(latex(derivative(f,x))) > 50) or (len(f.variables()) < 2):
    f = random_expression(randint(1,3))
  
\end{sagesilent}

\begin{exercise}

  Compute $\frac{\partial}{\partial\sage{x}} \left( \sage{f} \right)$.

  \begin{prompt}
    \[
      \frac{\partial}{\partial\sage{x}} \left( \sage{f} \right) = \answer{\sage{derivative(f,x)}}.
    \]
  \end{prompt}
  
\end{exercise}
\end{document}


%%% Local Variables:
%%% mode: latex
%%% TeX-master: t
%%% End:
