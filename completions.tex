\documentclass{ximera}

\title{Completions}

\begin{document}
\begin{abstract}
  Completions
\end{abstract}
\maketitle

\begin{example}
  \begin{question}
    \begin{multipleChoice}
      \choice[correct]{right}
      \choice{wrong}
    \end{multipleChoice}
    
    \begin{feedback}[correct]
      \begin{question}
        \begin{multipleChoice}
          \choice[correct]{rkight}
          \choice{wrong}
        \end{multipleChoice}

        \begin{feedback}[correct]
          Notice that in this case we have only one way to answer the
          question affirmatively since the solution is unique.
        \end{feedback}
      \end{question}
    \end{feedback}
  \end{question}

  \begin{question}
    \begin{multipleChoice}
      \choice[correct]{right}
      \choice{wrong}
    \end{multipleChoice}

    \begin{feedback}[correct]
      Notice that in this case we again have only one way to answer the
      question affirmatively since the solution is again unique.
    \end{feedback}
  \end{question}

  \begin{question}
    A question about membership in inevitably leads to a
    system of \wordChoice{\choice{one}\choice{two}\choice[correct]{three}} equations in \wordChoice{\choice{one}\choice{two}\choice[correct]{three}} variables
    $\alpha_1,\,\alpha_2,\,\alpha_3$ with a coefficient matrix whose

    This particular coefficient matrix is
    \wordChoice{\choice{singular}\choice[correct]{nonsingular}}, so by
    \ref{theorem:NMUS}, the system is guaranteed to have a solution.
    (This solution is unique, but that is not critical here.)  So
    \textit{no matter} which vector we might have chosen for
  \end{question}
\end{example}

\end{document}
