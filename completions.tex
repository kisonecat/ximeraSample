\documentclass{ximera}

\title{Completions}

\begin{document}
\begin{abstract}
  Completions
\end{abstract}
\maketitle

\begin{example}[A basic span]

  $7 = \answer{7}$
  $0 = \answer{0}$

  \begin{question}
    \begin{multipleChoice}
      \choice[correct]{Yes.}
      \choice{wrong}
    \end{multipleChoice}
    
    \begin{feedback}[correct]
      $\answer{5} = 5$
    \end{feedback}
  \end{question}
  
  \begin{question}
    Question?

    \begin{hint}
      hint
    \end{hint}
    
    \begin{multipleChoice}
      \choice{wrong.}
      \choice[correct]{right.}
    \end{multipleChoice}

    \begin{feedback}[correct]
      $\answer{2} = 2$
    \end{feedback}
  \end{question}
\end{example}

\begin{example}
  $2 = \answer{2}$

  \begin{question}
    \begin{multipleChoice}
      \choice[correct]{Yes.}
      \choice{wrong}
    \end{multipleChoice}

    \begin{feedback}[correct]
      another?
      \begin{multipleChoice}
        \choice[correct]{correct}
        \choice{wrong}
      \end{multipleChoice} 
    \end{feedback}
  \end{question}

  \begin{question}
    \[
      \begin{bmatrix}
        1 & 0 & 3 & \answer{3}\\
        0 & 1 & 2 & 2\\
        0 & 0 & 0 & 0
      \end{bmatrix}
    \]
    This system has \wordChoice{\choice[correct]{right}\choice{wrong}} (there is a free
    variable in $x_3$), but all we need is one solution vector.

    \begin{multipleChoice}
      \choice[correct]{right}
      \choice{wrong}
    \end{multipleChoice} 
  \end{question}

    
  \begin{question}
    \begin{multipleChoice}
      \choice{wrong.}
      \choice[correct]{right.}
    \end{multipleChoice}

    \begin{feedback}[correct]
      more
    \end{feedback}
  \end{question}

\end{example}

\begin{example}
  \begin{question}
    \begin{multipleChoice}
      \choice[correct]{right}
      \choice{wrong}
    \end{multipleChoice}
    
    \begin{feedback}[correct]
      \begin{question}
        \begin{multipleChoice}
          \choice[correct]{rkight}
          \choice{wrong}
        \end{multipleChoice}

        \begin{feedback}[correct]
          Notice that in this case we have only one way to answer the
          question affirmatively since the solution is unique.
        \end{feedback}
      \end{question}
    \end{feedback}
  \end{question}

  \begin{question}
    \begin{multipleChoice}
      \choice[correct]{right}
      \choice{wrong}
    \end{multipleChoice}

    \begin{feedback}[correct]
      so we are convinced that $\vect{x}$ really is in $\spn{R}$.
      Notice that in this case we again have only one way to answer the
      question affirmatively since the solution is again unique.
    \end{feedback}
  \end{question}

  \begin{question}
    A question about membership in $\spn{R}$ inevitably leads to a
    system of \wordChoice{\choice{one}\choice{two}\choice[correct]{three}} equations in \wordChoice{\choice{one}\choice{two}\choice[correct]{three}} variables
    $\alpha_1,\,\alpha_2,\,\alpha_3$ with a coefficient matrix whose
    columns are the vectors $\vect{v}_1,\,\vect{v}_2,\,\vect{v}_3$.

    This particular coefficient matrix is
    \wordChoice{\choice{singular}\choice[correct]{nonsingular}}, so by
    \ref{theorem:NMUS}, the system is guaranteed to have a solution.
    (This solution is unique, but that is not critical here.)  So
    \textit{no matter} which vector we might have chosen for
    $\vect{z}$, we would have been \textit{certain} to discover that
    it was an element of $\spn{R}$.  Stated differently, every vector
    of size 3 is in $\spn{R}$, or $\spn{R}=\complex{3}$.
  \end{question}
\end{example}

\end{document}
