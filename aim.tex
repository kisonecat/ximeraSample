\documentclass{ximera}

\title{MathJax madness}
\author{Jim Fowler}

\begin{document}
\begin{abstract}
  \LaTeX\ can be abused to do confusing things.
\end{abstract}

\maketitle

\begin{problem}
  Some calculus is
  \[
    \frac{d}{dx} \left( 2x^2 \right) = \answer{4x}.
  \]

  You can put \verb|\answer| in many places
    like \(\sqrt{\answer{9}} = 3\)
    or \[
      \frac{\answer{\sin \theta}}{\cos \theta} = \tan \theta.
    \]
\end{problem}

\begin{problem}
  A sage demo is
  \[
    \frac{d}{dx} \left( \sage{x*x + x*x} \right) =
      \answer{\sage{derivative(2*x^2,x)}}.
  \]
\end{problem}

\end{document}

