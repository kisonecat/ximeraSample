\documentclass{ximera}

\author{Jim Fowler}

\outcome{Compute the limit of vector-valued functions involving polynomial expressions.}

\begin{document}

\begin{sagesilent}
  p = -1
  t = var('t')
  
  xt = randint(0,2)*t*t + randint(0,2)*t + randint(0,2)
  yt = randint(0,2)*t*t + randint(0,2)*t + randint(0,2)
  zt = randint(0,2)*t*t + randint(0,2)*t + randint(0,2)
\end{sagesilent}

\begin{exercise}

  Let $\vec{f}(t) = \langle \sage{xt}, \sage{yt}, \sage{zt} \rangle$.

  Compute $\lim_{t \to \sage{p}} \vec{f}(t)$.

  \[
    \lim_{t \to \sage{p}} \vec{f}(t) = \langle \answer{\sage{limit(xt,t=p)}}, \answer{\sage{limit(yt,t=p)}}, \answer{\sage{limit(zt,t=p)}} \rangle.
  \]
  

    We compute the limit of a vector-valued function componentwise.



    Note that $\lim_{t \to \sage{p}} \left( \sage{xt} \right) = \sage{limit(xt,t=p)}$.



    Further $\lim_{t \to \sage{p}} \left( \sage{yt} \right) = \sage{limit(yt,t=p)}$.



    Also $\lim_{t \to \sage{p}} \left( \sage{zt} \right) = \sage{limit(zt,t=p)}$.

  

    Consequently $\lim_{t \to \sage{p}} \vec{f}(t) = \langle \sage{limit(xt,t=p)}, \sage{limit(yt,t=p)}, \sage{limit(zt,t=p)} \rangle$.

  
\end{exercise}
\end{document}


%%% Local Variables:
%%% mode: latex
%%% TeX-master: t
%%% End:
