\documentclass{ximera}

\title{Derivative flashcards}

\begin{document}

\begin{exercise}
Consider the indefinite integral:
 
\[
\int \frac{2x}{x^2-4} dx.
\]
 
This integral could be evaluated with a substitution, a trigonometric substitution, or by using partial fraction decomposition.
 
%%%%%%%%u-sub%%%%%%%%%%%%
If we want to evaluate the indefinite integral by using a substitution, let $u= \answer{x^2-4}$ so $du = \answer{2x} dx$.
 
\begin{exercise}
Making this substitution, we have:
 
\[
\int \frac{2x}{x^2-4} dx = \int \answer{\frac{1}{u}} du
\]
\begin{exercise}
Evaluating gives:
\[
\int \frac{2x}{x^2-4} dx = \int\frac{1}{u} du = \answer{\ln|u|+C}
\]
(type the antiderivative in terms of $u$ and use $C$ for the constant of integration)
 
and reversing the substitution gives:
 
\[
\int \frac{2x}{x^2-4} dx = \answer{\ln|x^2-4|+C}
\]
(type the antiderivative in terms of $x$ and use $C$ for the constant of integration)
\end{exercise}
\end{exercise}
%%%%%%%%%%%   Trig Sub  %%%%%%%%%%%%
\begin{exercise}
If we want to use a trigonometric substitution, we should set $u=\answer{2} \sec(\theta)$.  Then, $du = \answer{2 \sec(\theta)\tan(\theta)}$, and after substituting into the original integral and simplifying:
 
\[
\int \frac{2x}{x^2-4} dx = \int \answer{\frac{2 \sec^2(\theta)}{\tan(\theta)}} d\theta
\]
(type your answer in terms of $\sec(\theta)$ and $\tan(\theta)$.
 
\begin{exercise}
We can evaluate this integral and find:
 
\[
 \int \frac{2 \sec^2(\theta)}{\tan(\theta)} d\theta = \answer{2 \ln|\tan(\theta)|+C}
\]
(Use $C$ for the constant of integration and make sure that your antiderivative is in terms of $\theta$)
 
\begin{exercise}
 
To reverse the substitution, we must write $\tan(\theta)$ in terms of $x$.  In fact, we find:
 
\[
\tan(\theta) = \answer{\frac{\sqrt{x^2-4}}{2}}
\]
 
\begin{exercise}
We can now express the antiderivatives in terms of $x$ and find:
 
\[
\int \frac{2x}{x^2-4} dx =  2 \ln|\tan(\theta)|+C = \answer{2 \ln \left| \frac{\sqrt{x^2-4}}{2} \right| +C}
\]
 
\begin{exercise}
Using the properties of logarithms, we can show
 
\begin{align*}
2 \ln \left| \frac{ \sqrt{x^2-4} }{2} \right| & = 2 \ln \left|(x^2-4)^{1/2} \right| - 2 \ln(2) \\
& = 2 \cdot \frac{1}{2} \ln \left| x^2-4 \right| - 2 \ln(2) \textrm{ since } \ln\left(a^b\right) = b \ln(a) \\
& =  \ln \left|x^2-4 \right| - 2 \ln(2)
\end{align*}
which differs from the antiderivative found using the $u$-substitution method by a constant!
 
\end{exercise}
\end{exercise}
\end{exercise}
\end{exercise}
\end{exercise}
 
\[
\frac{2x}{x^2-4} = \frac{2x}{(x-2)(x+2)} = \frac{\answer{1}}{x-2}+\frac{\answer{1}}{x+2}
\]
  
\end{exercise}

\end{document}
