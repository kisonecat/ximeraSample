\documentclass{ximera}

\title{Derivative flashcards}

\begin{document}

\begin{exercise}
Let $f(x) = x^2-1$.

\begin{javascript}
  a = Math.round(10*Math.random())
  b = Math.round(10*Math.random())
  c = Math.round(10*Math.random())  

  function numericSubsetEquality(answer) {
    var input = this;

    answer = answer.sort();
    try {
      input = JSON.parse(input
        .replace('{','[').replace('}',']')
        .replace('(','[').replace(')',']'));
    } catch (error) {
      return false;
    }
    
     return input.length==answer.length &&
       answer.every(function(v,i) { return v === input[i]});
  }
\end{javascript}

\begin{problem}
  $\js{new Set([a,b,c])}$ is equal to $\answer[validator=numericSubsetEquality([a,b,c]),format=string]{}$.
\end{problem}

We are calculating the value of $\displaystyle \int_{-1}^{2} f(x) \d x$.
 

We start by setting up the Riemann sum using $n$ rectangles using right endpoints.
What is $\Delta x$?
     
\[ \Delta x = \answer{3/n} \]

\begin{exercise}
  Which of the following gives our choice of sample points?
  \begin{multipleChoice}
    \choice[correct]{$x_k^* = -1 + \frac{3k}{n}$}
    \choice{$x_k^* = -1 + \frac{3(k-1)}{n}$}
    \choice{$x_k^* = -1 + \frac{3\left(k-\frac{1}{2}\right)}{n}$}
  \end{multipleChoice}
  
  \begin{exercise}
    Calculate $f(x_k^*) \Delta x$ in terms of $k$ and $n$.
    
    \[ f(x_k^*) \Delta x = \frac{\answer{27}k^2}{n^3}-\frac{\answer{18}k}{n^2} \]
  \end{exercise}
\end{exercise}
\end{exercise}

\end{document}
