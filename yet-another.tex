\documentclass{ximera}

\title{Derivative flashcards}

\begin{document}

\begin{sagesilent}
  vs = ['x','y','z','w']
  x = var(vs[randint(0,len(vs)-1)])

  Fs = ['f','g','h','F','G','H']
  F = var(Fs[randint(0,len(Fs)-1)])
  
  def random_expression(depth):
    if depth == 0:
      return [x,1,-1][randint(0,2)]
    else:
      possible = [
        lambda: random_expression(depth-1) + random_expression(depth-1),
        lambda: random_expression(depth-1) - random_expression(depth-1),
        lambda: random_expression(depth-1) * random_expression(depth-1),
        lambda: random_expression(depth-1) / random_expression(depth-1),
        lambda: sin(random_expression(depth-1)),
        lambda: cos(random_expression(depth-1)),
        lambda: exp(random_expression(depth-1))
      ]
      result = possible[randint(0,len(possible)-1)]()
      if result.is_zero():
        result = 1
      return result

  f = random_expression(4)
  while (derivative(f,x).is_zero()) or (len(latex(derivative(f,x))) > 40):
    f = random_expression(randint(1,4))
\end{sagesilent}

\begin{exercise}

  Suppose $\sage{F}(\sage{x}) = \sage{f}$.

  Then $\sage{F}'(\sage{x}) = \answer{\sage{derivative(f,x)}}$.

\end{exercise}

\begin{example}
  This is an example.
\end{example}

\begin{quote}
\textbf{Over \textcolor{red!50!black}{some region},
\textcolor{green!70!black!70!blue}{sum up} products of
\textcolor{purple!50!blue!90!black}{heights} and \textcolor{blue!70!green}{areas}.}
\end{quote}

\end{document}
