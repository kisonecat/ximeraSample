\documentclass{ximera}

\title{Derivative flashcards}

\begin{document}

\begin{example}
  This is an example.

  $1 = \answer{1}$.
  
  \begin{problem}
    And so $2 = \answer{2}$.
  \end{problem}

  \begin{problem}
    And so $3 = \answer{3}$.
  \end{problem}
  
  \begin{hint}
    HELLO!
  \end{hint}  
\end{example}

\begin{quote}
\textbf{Over \textcolor{red!50!black}{some region},
\textcolor{green!70!black!70!blue}{sum up} products of
\textcolor{purple!50!blue!90!black}{heights} and \textcolor{blue!70!green}{areas}.}
\end{quote}

\begin{problem}
  $-\frac{\pi}{2}$ is the problem.
  
  $\answer{-\frac{\pi}{2}}$ was unanswerable

and
big thing is $\answer{\frac{g^{100}}{500}}$

$\answer{\frac{-\pi}{2}}$ is answerable
\end{problem}

\begin{itemize}
\item hELLO
\item yes
\item goodbye
\end{itemize}
\begin{exploration}
    \begin{tabular}{l|l|ll}
      \textbf{Value 1} & \textbf{Value 2} & \textbf{Value 3} & \textbf{Value 4}\\ % <-- added & and content for each column
      $\alpha$ & $\beta$ & $\gamma$ & $\delta$ \\ % <--
      \hline
      1 & 1110.1 & a & e\\ 
      2 & 10.1 & b & f\\ 
      3 & 23.113231 & c & g\\ 
    \end{tabular}
    \end{exploration}
\end{document}
